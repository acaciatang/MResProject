\documentclass[11pt]{article} %set font size,document type
\usepackage{geometry} %set paper size, orientation, margins
\geometry{a4paper, portrait, margin=2cm}
\usepackage{setspace} %set line spacing
\onehalfspace % 1.5 spacing
\renewcommand\intextsep{0pt} % no extra separation between figures and text
\usepackage{enumitem} % to set spacing between list and text
\renewcommand{\familydefault}{\sfdefault} %set font
\usepackage{helvet} % latex has no arial, using helvet instead
\usepackage{titlesec} % to format headings
\titleformat*{\section}{\normalsize\bfseries} % make section headings same font size as normal text but bold
\titleformat*{\subsection}{\normalsize\bfseries} % make subsection headings same font size as normal text but bold
\titlespacing*{\section} % set line spacing for section headings
{0pt}{0.6ex}{0pt} % left: 0, before: 1.4 times of a line, after: 0 
\titlespacing*{\subsection} % set line spacing for subsection headings
{0pt}{0ex}{0pt}
\usepackage[style=apa,  citestyle=authoryear, uniquelist=false]{biblatex} % set citation style
\addbibresource{MResProj.bib} % set bibliography file
\usepackage{url} % make urls same font as everything else
\urlstyle{same}
\usepackage{graphicx} % to add figures
\graphicspath{ .{../Figures/} } % path where figures can be found
\usepackage{siunitx} % to output units
\usepackage{float} % control position of figures
 
\begin{document}
    \begin{titlepage}\centering
    \vspace*{\fill}
    \LARGE The gut-brain axis: will manipulations of the gut microbiome destabilise social behaviour in bees?\\
    \vspace{\baselineskip}
    \LARGE Supervisor: Dr Peter Graystock\\
    \vspace{\baselineskip}
    \normalsize Author: Acacia Tang\\
    \normalsize Date: {\today}
    \vspace*{\fill}
    \end{titlepage}

    \newpage
    \section{Introduction}
        Microbes within a niche make up a microbiome.
        Studying microbiomes on/in organisms led to the discovery of the "gut-brain axis",
        through which gut microbiota affect the host's central nervous system,
        with implications in learning and communication
        \parencite{archie2015social}.
        Historically, bee microbiome studies focused on agricultural impacts.
        However, beyond economic import,
        eusocial bees could be a model for how the gut-brain axis affects social behaviour,
        as they have well-studied behaviours and a simple yet consistent microbiome.
        Correlations between the bee microbiome and social behaviour have provided promising insight:
        while unrelated to visual learning
        \parencite{leger2020gut},
        colony membership is influenced by the microbiome
        \parencite{vernier2020gut},
        and bees' microbiomes change with social roles
        \parencite{jones2018gut}.
        To the author's best knowledge, no study has established a direct link between social behaviour and microbiome composition in bees.
      
        This study aims to bridge this gap by studying the impact of microbiome disturbance on \textit{Bombus terrestis} behaviour in a controlled lab environment,
        focusing on the following questions:
        \begin{enumerate}[noitemsep,topsep=0pt]
            \item Does manipulating the gut microbiome change the social network of \textit{B. terrestis}?
            \item What behavioural changes (if any) are observed following administration of antibiotics?
            \item When do changes to social network (if any) occur post-disturbance?
            \item Does manipulating the gut microbiome change the reproductive efficiency of the hive?
        \end{enumerate}
 
    \section{Proposed Methods}
        \subsection{Experimental Set-up}
            The set-up used will be similar to that of Crall et al
            \parencite*{crall2018neonicotinoid}.
            Each study unit will comprise of the nest structure and brood from one colony along with 30 bees from the same colony,
            each bee will be tagged with BEEtag, as by Crall et al
            \parencite*{crall2015beetag}.
            Excess bees will be maintained outside the study unit.
            Newly emerged bees from the study unit will be counted and removed daily to be maintained together with the excess bees.
            The study unit will consist of 2 connected areas (nesting and foraging) monitored by cameras from the top down.
            Nesting areas hold the nest structure and brood,
            and the foraging area provides access to a sucrose feeder
            into which antibiotics will be added, if applicable.
            To quantify the consumption of antibiotics,
            nectar reserves will be weighed daily.
            The experimental set-up will be put on a 12:12 light:dark cycle.
            Each day, recording will be done for 23h,
            and camera batteries will be changed twice, during the last 30 min of the light and dark cycle respectively.
 
        \subsection{Treatments and replicates}
            3 treatments will be carried out: a no treatment control,
            a control group receiving antibiotic followed by probiotic,
            and an experimental group receiving only antibiotics.

            4 recording periods will be carried out.
            During each recording period,
            a replicate of each treatment/control will be carried out,
            for a total of 12 hives observed.
            Each recording period will last 15 days.
            No manipulations will be applied for the first 2 days.
            50 U/mL penicillin-streptomycin will be administered to antibiotic and antibiotic-probiotic groups on day 3.
            This dosage does not influence bee mortality but is above the minimum inhibitory concentration for most bacteria
            \parencite{miernik2007influence}.
            After antibiotics are administered,
            the antibiotic-probiotic control group will receive a daily 5\si{\milli\liter} spray of bee microbes from homogenized guts of 10 excess bees from the same colony.
            
        \subsection{Sampling individuals for microbiome}
            To assess changes in the microbiome,
            samples will be taken periodically and stored at \ang{-80}C for future further analysis.
            The sampling will be conducted as below.
            Before the start of the study,
            5 bees from each hive will be frozen.
            Following antibiotic treatment,
            swabs of the floor of the nesting area will be taken daily.
            A 100\si{\micro\liter} sample will also be taken from each gut homogenate to be applied to the antibiotic-probiotic control.
            Finally, all remaining bees will be frozen at the end of the study.
            
        \subsection{Back-up plan}
        If the above proposed lab-based work cannot be carried out,
        the analysis below will be carried out on recordings of a similar study on the effect of imidacloprid on the social behaviour of bees.
 
        \subsection{Analysis}
            Data collected will be fed into BEEtag software
            \parencite{crall2015beetag} to identify bee's positions.
            Behaviour will be recognised using a machine learning method,
            derivative of that designed by Blut et al
            \parencite*{blut2017automated}.
            Training will focus on nesting behaviours, which were observed with a similar set-up but cannot be distinguished by simple algorithms
            \parencite{crall2018neonicotinoid}.
            Parallelly,
            encounters, foraging, moving and nursing behaviour will be defined as by Crall et al
            \parencite*{crall2018neonicotinoid}
            and tallied.
            The error rates
            and number of behaviours recognised by these two methods will be compared to determine the more accurate one.
            
            Comparisons between and within treatments will be done by descriptive and temporal methods of network analysis.
            Interactions between bumblebees have no donor/recipient but the number of interactions is potentially important,
            so a non-directional weighted network will be analysed.
            5 metrics (density, strength, betweenness range, modularity and diameter)
            will be compared between hives.
            Density is the proportion of possible links observed
            \parencite{sosa2020network},
            showing how much of their colony individual bees are interacting with.
            Strength is the weighted sum of all links
            \parencite{sosa2020network},
            and shows the strength of interactions between bees.
            Betweenness is the number of times an individual is found on the shortest link between two others
            \parencite{sosa2020network}.
            A greater range of betweenness shows some individuals are more important to the social fabric of the hive.
            Modularity measures clustering within the network
            \parencite{sosa2020network}.
            High modularity shows the hive is working more as smaller clusters of bees,
            as opposed to a whole.
            Diameter is the longest shortest link in a network
            \parencite{sosa2020network},
            and shows the efficiency of relaying information through the hive.
            To study temporal changes in network structure,
            data from a hive will be grouped temporally at specific intervals and compared.
            Appropriate intervals will be found with ORA-LITE
            \parencite{carley2014ora}.

            Differences in reproductive output and rate of recorded behaviours
            will be assessed using repeated-measures ANOVA.
            Bonferroni correction will be used to compensate for multiple comparisons.
        
    \section{Anticipated outputs and outcomes}
        Based on the influence of microbiome on nestmate recognition in honey bees
        \parencite{vernier2020gut}, we expect bumblebees to demonstrate evidence of a gut-brain axis.
        Antibiotics may alter kin recognition,
        causing less social behaviour and a less cohesive social network.
        Studies on the effect of pesticides find an immediate impact that gradually disappears.
        \parencite{crall2018neonicotinoid}.

 
    \section{Project Timeline}
    \begin{figure}[H]
        \includegraphics[width=16cm]{gantt}
        \centering
    \end{figure}
 
    \section{Budget}
    12 bumblebee colonies £90ea + 1 8TB External Harddrive £200 + Plasticware and consumables £60
    For a total of £1,340. Dr Graystock will pay for costs that exceed the CMEE budget. The harddrive is needed as ~8TB of data comprised of ~20GB individual files is expected. No better alternative is avaliable: OneDrive has a storage limit of 5TB and requires unfeasible online syncing, while research data store is more expensive for long-term storage.
    
 
    \newpage
    \printbibliography
 
    \newpage
    \centering
    \vspace*{\fill}
        I have seen and approved the proposal and the budget.\\[8ex]
    \noindent\begin{tabular}{ll}
        \makebox[2.5in]{\hrulefill} & \makebox[2.5in]{\hrulefill}\\
        Dr Peter Graystock & Date\\
        Primary Supervisor
    \end{tabular}
    \vspace*{\fill}
\end{document}