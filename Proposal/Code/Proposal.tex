\documentclass[11pt]{article}
\usepackage{geometry}
\geometry{a4paper, portrait, margin=2cm}
\usepackage{setspace}
\onehalfspace
%\usepackage{biblatex}
%\renewcommand{\familydefault}{\sfdefault}
%\usepackage{font_here}
\usepackage{graphicx}
\graphicspath{ .{../Figures/} }

\begin{document}
    \begin{titlepage}\centering
    \vspace*{\fill}
    \LARGE The gut-brain axis: will manipulations of the gut microbiome destabilise social behaviour in bees?\\
    \vspace{\baselineskip}
    \LARGE Supervisor: Dr Peter Graystock\\
    \vspace{\baselineskip}
    \normalsize Author: Acacia Tang\\
    \normalsize Date: {\today}
    \vspace*{\fill}
    \end{titlepage}
  
    \newpage
    \section{Introduction}
        Studies on the microbiomes have revealed their influence on a wide range of host conditions including health, behaviour, and evolution
        \cite{}.
        One important discovery within this field is termed the "gut-brain axis",
        through which gut microbiota have been shown to affect host central nervous system,
        to the extent of being implicated in disease
        \cite{},
        learning
        \cite{},
        and communication
        \cite{}.
        Following these developments,
        there have been an increasing number of studies on the significance of microbiomes in bees.
        This research has been focused on the impact of antibiotics on honey bees, since antibiotics are used to maintain commercial honey hives.
        However, beyond its economical significance, this research has the potential to promote bees as a model for the gut-brain axis as it pertains to social behaviour,
        since bees are eusocial species with well-studied and documented behaviours.
        A few studies on the effect of microbiome on the social behaviour of bees to date.
        The results from these are promising: while no link was found between microbiome and visual learning
        \cite{leger2020gut},
        but there is evidence that colony membership is determined by microbiome composition
        \cite{vernier2020gut},
        and there are known differences in microbiome between bees with different social roles
        \cite{jones2018gut}.
        However, to the authour's best knowledge no study has established a direct link between social behaviour and microbiome composition in bees.
        
        This study aims to bridge this gap by monitoring the behaviour of bumblebees in artificial hives and investigating the impact of microbiome on their behaviour,
        focusing on the following three questions:
        \begin{enumerate}
            \item Does administration of antibiotics change the social network of bumblebees?
            \item What behavioural changes (if any) are observed following administration of antibiotics?
            \item When do changes (if any) occur?
        \end{enumerate}

    \section{Proposed Methods}
        \subsection{Experimental Set-up}
            Experimental set-up will be similar to that outlined by Crall et al
            \cite{crall2018neonicotinoid}.
            In brief, bees will be tagged with BEEtag following the methods put forth by Crall et al
            \cite{crall2015beetag}.

        \subsection{Treatments and replicates}
            Three treatments will be carried out: no treatment control,
            control group receiving antibiotic followed by probiotic,
            and experimental group receiving only antibiotic.
            During each 2 week recordining period,
            two replicates of each treatment will be carried out.
            Three recording periods will be carried out in total.

        \subsection{Data recording}
            Data recording will be done using a camera at 4 frames for second. The data collected will be in the format of 
            
        \subsection{Analysis}
            Behaviour will be distinguished using machine learning, using a similar method as the machine learning people
            \cite{blut2017automated}.

    \section{Anticipated outputs and outcomes}
        Based on a recent study on the influence of microbiome on CHC production and nestmate recognition in honey bees
        \cite{vernier2020gut},
        it is expected that antibiotic treatment may 
        Change in mortality is immediately noticeable
        \cite{li2017new}, but it is possible that behaviour will take longer.

    \section{Project Timeline}
        The proposed project timeline is as below:
        \begin{figure}[h]
            \includegraphics[width=17cm]{gnatt}
            \centering
        \end{figure}

    \section{Budget}
        High-Performance Computing time?

    \newpage
    \bibliographystyle{plain}
    \bibliography{MResProj}

    \newpage
    \centering
    \vspace*{\fill}
        I have seen and approved the proposal and the budget.\\[8ex]
    \noindent\begin{tabular}{ll}
        \makebox[2.5in]{\hrulefill} & \makebox[2.5in]{\hrulefill}\\
        Dr Peter Graystock & Date\\
        Primary Supervisor
    \end{tabular}
    \vspace*{\fill}
\end{document}