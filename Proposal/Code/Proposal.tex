\documentclass[11pt]{article}
\usepackage{geometry}
\geometry{a4paper, portrait, margin=2cm}
\usepackage{setspace}
\onehalfspace
\usepackage{bibtex}
\bibliographystyle{apalike}


%\usepackage{biblatex}
%\renewcommand{\familydefault}{\sfdefault}
%\usepackage{font_here}
\usepackage{graphicx}
\graphicspath{ .{../Figures/} }
\usepackage{siunitx}
\usepackage{float}
 
\begin{document}
    \begin{titlepage}\centering
    \vspace*{\fill}
    \LARGE The gut-brain axis: will manipulations of the gut microbiome destabilise social behaviour in bees?\\
    \vspace{\baselineskip}
    \LARGE Supervisor: Dr Peter Graystock\\
    \vspace{\baselineskip}
    \normalsize Author: Acacia Tang\\
    \normalsize Date: {\today}
    \vspace*{\fill}
    \end{titlepage}

    \newpage
    \section{Introduction}
        The microbiome is a collective term for all the microbes that live within a specific niche.
        Studies on the microbiomes found on and in organisms have,
        in particular,
        revealed their influence on a wide range of host conditions including health, behaviour, and evolution
        \cite{archie2015social}.
        One important discovery within this field is termed the "gut-brain axis",
        through which gut microbiota have been shown to affect host central nervous system,
        to the extent of being implicated in disease, learning, and communication
        \cite{archie2015social}.
        There has been on-going interest in the significance of microbiomes in bees,
        but past research has been concerned mainly with nutrition, defense against parasites, etc.
        This research has also been focused on the impact of antibiotics (especially tetracyclines) on honey bees,
        as these are used to maintain commercial honey hives.
        However, beyond economic import,
        research into the significance of the bee microbiome has the potential to promote bees as a model for the gut-brain axis as it pertains to social behaviour,
        since bees are eusocial species with well-studied and documented behaviours.
        Several studies on correlations between microbiome and social behaviour have been reported in bees.
        The results from these have been promising: while no link was found between microbiome and visual learning
        \cite{leger2020gut},
        there is evidence that colony membership is determined by microbiome composition
        \cite{vernier2020gut},
        and there are known differences in microbiome between bees with different social roles
        \cite{jones2018gut}.
        However, to the author's best knowledge no study has established a direct link between social behaviour and microbiome composition in bees.
      
       This study aims to bridge this gap by monitoring the behaviour of \textit{Bombus terrestis} in a controlled lab environment and investigating the impact of microbiome disturbance on their behaviour,
       focusing on the following questions:
       \begin{enumerate}
            \item Does manipulating the gut microbiome change the social network of \textit{B. terrestis}?
            \item What behavioural changes (if any) are observed following administration of antibiotics?
            \item When do changes to social network (if any) occur post-disturbance?
            \item When do changes to individual behaviour (if any) occur post-disturbance?
            \item Does manipulating the gut microbiome change the reproductive efficiency of the hive?
       \end{enumerate}
 
    \section{Proposed Methods}
        \subsection{Experimental Set-up}
            Experimental set-up will be similar to that outlined by Crall et al
            \cite{crall2018neonicotinoid}.
            In brief,
            each study unit will comprise of the nest structure and brood from one colony along with 40 bees from the same colony,
            each tagged with BEEtag following the methods put forth by Crall et al
            \cite{crall2015beetag}.
            The rest of the hive will be left unmanipulated and maintained outside the study unit.
            Newly emerged bees from the study unit will be removed daily and maintained with the rest of the hive.
            The experimental set-up will consist of two connected areas:
            one for nesting and one for foraging.
            The nesting area will hold the nest structure and brood removed from the colony,
            and the foraging area will provide access to a nectar reserve,
            into which antibiotics will be added if treatment involves administration of antibiotics.
            To quantify the consumption of antibiotics,
            the nectar reserve will be weighed daily.
            The experimental set-up will be put on a 12:12 light:dark cycle.
 
        \subsection{Treatments and replicates}
            Three treatments will be carried out: a no treatment control,
            a control group receiving antibiotic followed by probiotic,
            and an experimental group receiving only antibiotics.

            Four recording periods will be carried out in total.
            During each recording period,
            two replicates of each treatment/control will be carried out,
            for a total of 6 hives observed per period.

            Each recording period will last 15 days.
            For the first two days, no manipulations will be applied.
            Then, a spike of 50X diluted penicillin-streptomycin will be administered to antibiotic and antibiotic-probiotic groups on the third day.
            This dosage was used by Li et al
            \cite{li2017new},
            who found a 100\% mortality within 14 days of chronic exposure.
            After antibiotics are administered,
            the antibiotic-probiotic control group will receive a daily spray of homogenized guts from bees belonging to the same colony at a concentration of \(10^9\)
            colony-forming units per mL for the remainder of the recording period.
            
       \subsection{Data recording}
            Data for each artificial hive will be recorded using two cameras affixed above the nesting and foraging area.
            Recording will be done for 23h a day at 4 frames per second, 
            the frame rate used by Blut et al
            \cite{blut2017automated}.
            Two battery changes will be conducted a day,
            during the last 30 min of the light and dark cycle respectively.

            The number of newly emerged bees will be manually counted and removed from the hive daily.
        
        \subsection{Sampling individuals for microbiome}
            To provide a snapshot of the changes in the microbiome, bees will be sampled periodically and washed with 12.5\% bleach solution and water,
            then flash-frozen and stored at \ang{-80}C for future further analysis,
            as was the method used by Vernier et al
            \cite{vernier2020gut}.
            The sampling will be conducted as below. Before the start of the experiment, one bee from each hive will be frozen.
            During the ten days of data recording following the antibiotic treatment, one bee from each colony will be removed and frozen.
            Finally, at the end of the experiment, all bees remaining will be frozen.

            In addition to this, a 100\si{\micro\liter} sample from each gut homogenate to be applied to the antibiotic-probiotic control will be frozen at \ang{-80}C.
            
        \subsection{Back-up plan}
        In the event that the above proposed methods are unable to be carried out,
        The analysis below will be carried out on recordings of a similar experiment investigating the effect of imidacloprid pesticide on the social behaviour of bees.
 
        \subsection{Analysis}
            The data collected will be feed into the BEEtag software
            \cite{crall2015beetag} to identify the position of bees.
            
            Behaviour will then be distinguished using machine learning,
            using a method derivative of that designed by Blut et al
            \cite{blut2017automated}.
            In parallel to machine learning,
            encounters, foraging, moving and nursing behaviour will be defined in the same way as Crall et al
            \cite{crall2018neonicotinoid},
            and tallied for each individual.
            The estimated false positive rate, estimated false negative rate,
            and number of different behaviours able to be distinguished using these two methods of data analysis will be compared to determine the more accurate method of behaviour determination.
            
            Comparisons between colonies will be carried out using network analysis,
            using descriptive, temporal, and spatial methods reviewed by Pinter et al
            \cite{pinter2014dynamics}.
            Since interactions between bumblebees (encounters) have no donor/recipient but the number of interactions between individuals is potentially important, a non-directional weighted network will be generated from the data.
            In brief, five metrics (density, strength, betweenness range, modularity and diameter) will be compared between hives.
            Density is the proportion of possible links that are observed
            \cite{sosa2020network},
            representing how much of their colony individual bees are interacting with.
            Strength is the weighted sum of all links
            \cite{sosa2020network},
            and represents the strength of interactions between bees.
            Betweenness is the number of times an individual is found along the shortest link between two other individuals
            \cite{sosa2020network}.
            The range of betweenness reflects the robustness of the social structure of the hive.
            A greater range reflects the presence of individuals which are more important than others for maintaining the social fabric of the hive.
            Modularity is a measure of clustering within the social network
            \cite{sosa2020network}.
            High modularity indicates that the hive is working more as smaller clusters of bees,
            while low modularity shows that the hive is working more as a whole.
            Diameter is the longest shortest link between two individuals in a network
            \cite{sosa2020network},
            which is a measure of the efficiency of relaying information through the hive.

            To investigate temporal changes in social network structure, a discrete approach will be used,
            where data from the same hive is grouped temporally at specific intervals and compared.
            The appropriate intervals will be determined using ORA-LITE
            \cite{carley2014ora}.

            The presence of change(s) in reproductive output (used to quantify the significance of any changes observed in social behaviour on the hive as a whole, and measured by number of newly emerged bees)
            and in rate of exhibition of each recorded behaviour within each treatment
            will be assessed using repeated-measures ANOVA.
            To compensate for multiple comparisons, a Bonferroni correction will be applied to the results of ANOVA.
        
    \section{Anticipated outputs and outcomes}
        Based on a recent study on the influence of microbiome on CHC production and nestmate recognition in honey bees
        \cite{vernier2020gut},
        it is expected that antibiotic treatment may alter kin recognition in bumblebees as well,
        leading to less social behaviour.
        Potentially, rejection of nestmates
        \cite{vernier2020gut} may also be observed.
        No studies to date have carried out any manipulations studies to investigate the effect of microbiome on social behaviour in bees,
        but studies of the effect of pesticides on honey bee behaviours found an immediate impact followed by a gradual decrease in difference between the treatment and control groups.
        \cite{crall2018neonicotinoid}.

 
    \section{Project Timeline}
        The proposed project timeline is as below:
        \begin{figure}[H]
            \includegraphics[width=17cm]{gnatt}
            \centering
        \end{figure}
 
    \section{Budget}
        [insert lies here]
 
    \newpage
    
    \bibliography{MResProj}
 
    \newpage
    \centering
    \vspace*{\fill}
        I have seen and approved the proposal and the budget.\\[8ex]
    \noindent\begin{tabular}{ll}
        \makebox[2.5in]{\hrulefill} & \makebox[2.5in]{\hrulefill}\\
        Dr Peter Graystock & Date\\
        Primary Supervisor
    \end{tabular}
    \vspace*{\fill}
\end{document}