\documentclass[11pt]{article}
\usepackage{geometry}
\geometry{a4paper, portrait, margin=2cm}
\usepackage{setspace}
\onehalfspace
\usepackage{biblatex}
\renewcommand{\familydefault}{\sfdefault}
\usepackage{font_here}

\begin{document}

    \begin{titlepage}\centering
    \vspace*{\fill}
    \LARGE The gut-brain axis: will manipulations of the gut microbiome destabilise social behaviour in bees?\\
    \vspace{\baselineskip}
    \LARGE Supervisor: Dr Peter Graystock\\
    \vspace{\baselineskip}
    \normalsize Author: Acacia Tang\\
    \normalsize Date: {\today}
    \vspace*{\fill}
    \end{titlepage}
  
    \newpage
    \section{Introduction}
        Studies on the microbiomes have revealed an influence on a range of host conditions including health, behaviour, and evolution
        \cite{}.
        One important discovery within this field is termed the "gut-brain axis",
        through which gut microbiota have been shown to affect host central nervous system,
        to the extent of being implicated in disease
        \cite{},
        learning
        \cite{},
        and communication
        \cite{}.
        Following these developments,
        there have been an increasing number of studies on the significance of microbiomes in bees.
        This research has been focused on the impact of antibiotics on honey bees, since antibiotics are used to maintain commercial honey hives.
        Beyond its economical significance, this research has the potential as a model for the gut-brain axis as it pertains to social behaviour,
        since bees are eusocial species with well-studied and documented behaviours.
        Despite this, there are few studies on the effect of microbiome on the social behaviour of bees to date.
        The results from these are mixed: there appears to be no link between microbiome and visual learning
        \cite{leger2020gut},
        but there is evidence that colony membership is determined by microbiome composition
        \cite{vernier2020gut},
        and there are known differences in microbiome between bees that carry out different tasks
        \cite{jones2018gut}.
        

    \section{Proposed Methods}
        \subsection{Experimental Set-up}
            Experimental set-up will be similar to that Crall et al
            \cite{crall2018neonicotinoid},
            Bees will be tagged with BEEtag
            \cite{crall2015beetag}.

        \subsection{Treatments and replicates}
            Three treatments: NTC, antibiotic, antibiotic and species of lactobacillus
            \cite{ptaszynska2016commercial}. 

        \subsection{Data recording}
            Data recording will be done using a camera at 4 frames for second.
            
        \subsection{Analysis}
            Behaviour will be distinguished using machine learning, using a siimilar method as the machine learning people
            \cite{blut2017automated}.

    \section{Anticipated outputs and outcomes}
        Based on the CHC paper
        \cite{vernier2020gut}, I expect the bees to be less social.
        Change in mortality is immediately notiable
        \cite{li2017new}, but it is possible that behaviour will take longer.

    \section{Project Timeline}
        I have 8 months. The first will be used to set-up everything and develop a pipeline for data-analysis.
        Then will be two months of data-collection, with a month of buffer.
        Pipeline can be continued to be made during this time.
        Then we have three months of data analysis and finally one month of writing.

    \section{Budget}
        High-Performance Computing time?

    \newpage
    \bibliographystyle{}
    \bibliography{MResProj}

    \newpage
    \vspace*{\fill}
        I have seen and approved the proposal and the budget.\\[8ex]
    \noindent\begin{tabular}{ll}
        \makebox[2.5in]{\hrulefill} & \makebox[2.5in]{\hrulefill}\\
        Dr Peter Graystock & Date\\
        Primary Supervisor
    \end{tabular}
    \vspace*{\fill}
\end{document}